%%%%%%%%%%%%%%%%%%%%%%%%%%%%%%%%%%%%%%%%%%%%%%%%%%%%%%%%%%%%%%%%%
%%% %
%%% % weiiszablon.tex
%%% % The Faculty of Electrical and Computer Engineering
%%% % Rzeszow University Of Technology diploma thesis Template
%%% % Szablon pracy dyplomowej Wydziału Elektrotechniki 
%%% % i Informatyki PRz
%%% % January, 2024
%%%%%%%%%%%%%%%%%%%%%%%%%%%%%%%%%%%%%%%%%%%%%%%%%%%%%%%%%%%%%%%%%

\documentclass[12pt,twoside]{article}

\usepackage{weiiszablon}

\author{Kacper Rychel}

% np. EF-123456, EN-654321, ...
\studentID{EF-173701}

\title{Robot mobilny z nawigacją autonomiczną oparty na Arduino}
\titleEN{Mobile robot with autonomous navigation based on Arduino}


%%% wybierz rodzaj pracy wpisując jeden z poniższych numerów: ...
% 1 = inżynierska	% BSc
% 2 = magisterska	% MSc
% 3 = doktorska		% PhD
% 4 = praca inżynierska
%%% na miejsce zera w linijce poniżej
\newcommand{\rodzajPracyNo}{1}


%%% promotor
\supervisor{Dr inż. Mariusz Mączka}
%% przykład: dr hab. inż. Józef Nowak, prof. PRz

%%% promotor ze stopniami naukowymi po angielsku
\supervisorEN{Prof. Mariusz Mączka}

\abstract{Cel pracy: Stworzenie robota mobilnego, który potrafi omijać przeszkody i poruszać się autonomicznie w środowisku. Zakres pracy: zaprojektowanie platformy robota z silnikami i czujnikami ultradźwiękowymi do detekcji przeszkód; zaprogramowanie algorytmu omijania przeszkód i nawigacji w środowisku; możliwość rozbudowy o funkcje śledzenia wyznaczonego celu lub powrotu do bazy. Testy robota w różnych scenariuszach, takich jak omijanie przeszkód czy poruszanie się po określonej trasie.}
\abstractEN{Objective: To create a mobile robot that can avoid obstacles and navigate autonomously within its environment. Scope of work: Designing a robot platform with motors and ultrasonic sensors for obstacle detection; programming an algorithm for obstacle avoidance and navigation within the environment; and the potential for expansion with target tracking or return-to-base functionality. Testing the robot in various scenarios, such as obstacle avoidance and navigating along a defined route.}

\keywords{Arduino Uno, Robot mobilny, Czujnik HC-SR04, Czujnik Ultradźwiękowy Odległości}
\keywordsEN{Arduino Uno, Mobile robot, HC-SR04 sensor, Ultrasonic distance sensor}


\begin{document}

% strona tytułowa
\maketitle

\blankpage

% spis treści
\tableofcontents

\clearpage
\blankpage

\section{Wprowadzenie}
Roboty mobilne stanowią obecnie jeden z najintensywniej rozwijających się obszarów robotyki oraz inżynierii mechatronicznej. Ich rosnące znaczenie wynika z szerokiego spektrum zastosowań obejmujących przemysł, logistykę, eksplorację środowisk niebezpiecznych, systemy autonomiczne oraz edukację techniczną. Zgodnie z definicją Międzynarodowej Federacji Robotyki robot mobilny jest systemem mechatronicznym zdolnym do samodzielnego przemieszczania się w przestrzeni oraz realizacji określonych zadań bez stałego połączenia z infrastrukturą stacjonarną [1]. Współczesny rozwój technologii mikroprocesorowych oraz dostępność platform open-source znacząco ułatwiły projektowanie i implementację tego typu systemów. Jak zauważa Monk, popularyzacja mikrokontrolerów, takich jak Arduino, umożliwiła szybkie prototypowanie rozwiązań robotycznych nawet w warunkach dydaktycznych i amatorskich [2].

Kluczowym zagadnieniem w robotyce mobilnej jest autonomia, rozumiana jako zdolność robota do samodzielnego podejmowania decyzji na podstawie danych pochodzących z czujników. Autonomiczne systemy mobilne muszą integrować warstwy percepcji, planowania oraz sterowania ruchem. Według Siegwart’a, Nourbakhsha i Scaramuzzy autonomia robota opiera się na ścisłej współpracy tych trzech elementów, nawet w przypadku prostych algorytmów reaktywnych [3]. W konstrukcjach edukacyjnych i prototypowych stosuje się zazwyczaj uproszczone algorytmy decyzyjne, których celem jest zapewnienie podstawowych funkcji, takich jak unikanie przeszkód czy poruszanie się w nieznanym środowisku.

Istotnym elementem systemów autonomicznych są czujniki umożliwiające detekcję otoczenia. W robotach niskokosztowych powszechnie stosowane są czujniki ultradźwiękowe, które wykorzystują zjawisko odbicia fali akustycznej do pomiaru odległości od przeszkody. Czujnik HC-SR04, często wykorzystywany w projektach edukacyjnych, umożliwia pomiar odległości w zakresie od kilku do kilkuset centymetrów przy stosunkowo niewielkim koszcie oraz prostej integracji z mikrokontrolerem [4]. Zgodnie z dokumentacją techniczną producenta, dokładność pomiaru jest wystarczająca dla podstawowych algorytmów nawigacyjnych, choć należy uwzględnić podatność czujnika na zakłócenia wynikające z charakterystyki powierzchni odbijających falę ultradźwiękową [5].

Ruch robota mobilnego realizowany jest najczęściej przy wykorzystaniu napędu różnicowego, składającego się z dwóch niezależnie sterowanych silników prądu stałego. Taki układ umożliwia zmianę kierunku jazdy oraz skręcanie poprzez regulację prędkości obrotowej poszczególnych kół. Jak wskazują Jones, Flynn i Seiger, napęd różnicowy jest jednym z najprostszych i najbardziej rozpowszechnionych rozwiązań w robotach mobilnych ze względu na niewielką złożoność mechaniczną i sterowniczą [6]. Do sterowania silnikami prądu stałego stosuje się układy typu mostek H, takie jak L293D, które pozwalają na zmianę kierunku obrotów oraz regulację prędkości za pomocą sygnału modulowanego szerokością impulsu PWM. Horowitz i Hill podkreślają, że mostki H stanowią podstawowy element układów wykonawczych w systemach sterowania napędem o małej mocy [7].

Centralnym elementem sterującym robota mobilnego jest mikrokontroler, który odpowiada za przetwarzanie danych sensorycznych oraz generowanie sygnałów sterujących elementami wykonawczymi. Platforma Arduino UNO, oparta na mikrokontrolerze ATmega328P, jest jedną z najczęściej wykorzystywanych w projektach edukacyjnych i badawczych ze względu na dostępność dokumentacji, bibliotek programistycznych oraz wsparcie społeczności open-source [2][8]. Zastosowanie gotowych bibliotek umożliwia szybkie wdrażanie algorytmów sterowania ruchem oraz obsługi czujników bez konieczności implementowania niskopoziomowych mechanizmów sprzętowych.

Projekty polegające na budowie autonomicznych robotów mobilnych mają istotne znaczenie dydaktyczne, gdyż umożliwiają integrację wiedzy z zakresu elektroniki, programowania, automatyki oraz mechaniki w jednym, spójnym systemie. Jak zauważa Corke, robotyka stanowi naturalne środowisko do kształcenia kompetencji inżynierskich, łącząc teorię z praktycznym rozwiązywaniem rzeczywistych problemów technicznych [9]. Z tego względu realizacja projektu autonomicznego robota mobilnego stanowi wartościowe zagadnienie pracy dyplomowej o charakterze inżynierskim.

Rozwój autonomicznych robotów mobilnych oraz ich rosnące znaczenie w edukacji i przemyśle stanowi bezpośrednie uzasadnienie wyboru tematyki pracy. Celem niniejszej pracy jest zaprojektowanie i wykonanie autonomicznego robota mobilnego opartego na mikrokontrolerze Arduino UNO, zdolnego do samodzielnego poruszania się oraz unikania przeszkód na podstawie pomiarów odległości. Zakres pracy obejmuje analizę teoretyczną zagadnień związanych z robotyką mobilną, projekt i realizację układu sprzętowego, implementację algorytmu sterowania oraz przeprowadzenie testów funkcjonalnych opracowanego rozwiązania.

\section{Przegląd stanu wiedzy i techniki w zakresie robotów mobilnych}
Rozwój robotów mobilnych w ostatnich dekadach doprowadził do wykształcenia się licznych podejść konstrukcyjnych i algorytmicznych, których dobór zależy od przeznaczenia robota, środowiska pracy oraz dostępnych zasobów sprzętowych. W literaturze przedmiotu podkreśla się, że mimo znacznego postępu w dziedzinie autonomii i sztucznej inteligencji, wciąż istotną rolę odgrywają proste, deterministyczne rozwiązania, szczególnie w systemach o charakterze edukacyjnym i prototypowym [10].

Jednym z podstawowych zagadnień w robotyce mobilnej jest modelowanie ruchu robota. W przypadku najczęściej stosowanego napędu różnicowego ruch robota opisywany jest za pomocą kinematyki nieliniowej, w której prędkość liniowa oraz kątowa platformy wynikają bezpośrednio z prędkości obrotowych kół napędowych. Jak wskazuje Campion, Bastin i D’Andrea-Novel, model kinematyczny robota z napędem różnicowym jest prosty w implementacji, lecz jednocześnie wprowadza ograniczenia w postaci braku możliwości ruchu bocznego [11]. Ograniczenie to wymusza stosowanie odpowiednich strategii planowania trajektorii oraz manewrowania w środowiskach o ograniczonej przestrzeni.

W praktycznych realizacjach, szczególnie w systemach o niewielkiej mocy obliczeniowej, sterowanie ruchem robota realizowane jest w oparciu o regulatory dyskretne, często bez jawnego wykorzystania modelu dynamicznego. Zastosowanie modulacji szerokości impulsu PWM umożliwia płynną regulację prędkości silników prądu stałego, natomiast zmiana kierunku obrotów realizowana jest poprzez odpowiednie przełączanie mostka H. Jak zauważają Horowitz i Hill, takie podejście, mimo swojej prostoty, jest w pełni wystarczające dla systemów, w których nie jest wymagana wysoka precyzja pozycjonowania [7].

Istotnym elementem wpływającym na skuteczność autonomii robota mobilnego jest sposób przetwarzania danych sensorycznych. W systemach niskokosztowych dominują rozwiązania oparte na pojedynczych lub kilku czujnikach odległości, których wskazania wykorzystywane są bezpośrednio do podejmowania decyzji ruchowych. W literaturze określa się to podejście mianem nawigacji reaktywnej, w której robot nie buduje globalnej reprezentacji środowiska, lecz reaguje na aktualnie wykryte przeszkody [6]. Metody te, choć nieoptymalne z punktu widzenia długości trajektorii czy efektywności ruchu, cechują się dużą odpornością na zmiany otoczenia oraz niewielkimi wymaganiami obliczeniowymi.

Czujniki ultradźwiękowe, takie jak HC-SR04, są powszechnie wykorzystywane w tego typu algorytmach ze względu na prostotę zasady działania oraz łatwość integracji z mikrokontrolerami. Jak wskazują badania porównawcze czujników odległości, dokładność pomiarów ultradźwiękowych jest wystarczająca dla wykrywania przeszkód o rozmiarach porównywalnych z wymiarami robota, jednak maleje w przypadku powierzchni pochłaniających fale akustyczne lub ustawionych pod dużym kątem względem osi pomiaru [12]. Z tego względu w literaturze zaleca się stosowanie odpowiednich marginesów bezpieczeństwa w algorytmach decyzyjnych oraz filtrację wyników pomiarów [13].

W bardziej zaawansowanych systemach mobilnych stosuje się techniki lokalizacji i mapowania jednoczesnego (SLAM), wykorzystujące czujniki lidarowe, kamery lub systemy inercyjne. Metody te pozwalają na tworzenie map środowiska oraz precyzyjne określanie położenia robota, jednak wymagają znacznie większych zasobów obliczeniowych oraz złożonego oprogramowania [14]. Jak podkreśla Thrun, implementacja algorytmów SLAM w systemach edukacyjnych jest często nieuzasadniona ze względu na stopień skomplikowania, który może przesłaniać podstawowe zagadnienia inżynierskie [15].

Z tego względu w projektach dydaktycznych oraz pracach inżynierskich często koncentruje się na rozwiązaniach kompromisowych, łączących prostotę implementacji z funkcjonalnością umożliwiającą demonstrację kluczowych zagadnień robotyki mobilnej. Budowa autonomicznego robota mobilnego opartego na platformie Arduino, wyposażonego w czujniki ultradźwiękowe i napęd różnicowy, wpisuje się w ten nurt, umożliwiając praktyczną realizację algorytmów sterowania oraz analizę ich ograniczeń.

Podsumowując, aktualny stan wiedzy i techniki w zakresie robotów mobilnych obejmuje zarówno zaawansowane systemy autonomiczne stosowane w przemyśle i badaniach naukowych, jak i uproszczone rozwiązania edukacyjne, które odgrywają istotną rolę w procesie kształcenia inżynierów. Analiza literatury wskazuje, że nawet proste konstrukcje mobilne pozwalają na realizację kluczowych funkcji autonomii, takich jak percepcja otoczenia, podejmowanie decyzji oraz sterowanie ruchem, stanowiąc solidną podstawę do dalszego rozwoju bardziej złożonych systemów robotycznych.


\clearpage

\addcontentsline{toc}{section}{Literatura}

\begin{thebibliography}{4}
\bibitem{str} http://weii.portal.prz.edu.pl/pl/materialy-do-pobrania. Dostęp 5.01.2015.
\bibitem{Jakubczyk1997} Jakubczyk T., Klette A.: Pomiary w akustyce. WNT, Warszawa 1997.
\bibitem{Barski2011} Barski S.: Modele transmitancji. Elektronika praktyczna, nr 7/2011, str. 15-18.
\bibitem{dokum} Czujnik S200. Dokumentacja techniczno-ruchowa. Lumel, Zielona Góra, 2001.
\bibitem{Pawluk2001} Pawluk K.: Jak pisać teksty techniczne poprawnie, Wiadomości Elektrotechniczne, Nr 12, 2001, str. 513-515.
\end{thebibliography}

\clearpage

\makesummary

\end{document} 
